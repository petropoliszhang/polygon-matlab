\documentclass{article}
%%%%%%%%%%%%%%%%%%%%%%%%%%%%%%%%%%%%%%%%%%%%%%%%%%%%%%%%%%%%%%%%%%%%%%%%%%%%%%%%%%%%%%%%%%%%%%%%%%%%%%%%%%%%%%%%%%%%%%%%%%%%%%%%%%%%%%%%%%%%%%%%%%%%%%%%%%%%%%%%%%%%%%%%%%%%%%%%%%%%%%%%%%%%%%%%%%%%%%%%%%%%%%%%%%%%%%%%%%%%%%%%%%%%%%%%%%%%%%%%%%%%%%%%%%%%
\usepackage{amsmath,amssymb}
% more math
\usepackage{amsfonts}
\usepackage{amssymb}
\usepackage{amstext}
\usepackage{amsbsy}

\usepackage{color}
\newcommand{\mt}[1]{\marginpar{\small #1}}
%%%%%%%%%%%%%%%%%%%%%%%%%%%%%%%%%%%%%%%%%%%%%%%%%%%%%%%%%%%%%%%%%%%%
% new commands
\newcommand{\nc}{\newcommand}
% operators
\renewcommand{\div}{\vec{\nabla}\! \cdot \!}
\newcommand{\grad}{\vec{\nabla}}
% latex shortcuts
\newcommand{\bea}{\begin{eqnarray}}
\newcommand{\eea}{\end{eqnarray}}
\newcommand{\be}{\begin{equation}}
\newcommand{\ee}{\end{equation}}
\newcommand{\bal}{\begin{align}}
\newcommand{\eali}{\end{align}}
\newcommand{\bi}{\begin{itemize}}
\newcommand{\ei}{\end{itemize}}
\newcommand{\ben}{\begin{enumerate}}
\newcommand{\een}{\end{enumerate}}
% DGFEM commands
\newcommand{\jmp}[1]{[\![#1]\!]}                     % jump
\newcommand{\mvl}[1]{\{\!\!\{#1\}\!\!\}}             % mean value
\newcommand{\keff}{\ensuremath{k_{\textit{eff}}}\xspace}
% shortcut for domain notation
\newcommand{\D}{\mathcal{D}}
% vector shortcuts
\newcommand{\vo}{\vec{\Omega}}
\newcommand{\vr}{\vec{r}}
\newcommand{\vn}{\vec{n}}
\newcommand{\vnk}{\vec{\mathbf{n}}}
\newcommand{\vj}{\vec{J}}
% extra space
\newcommand{\qq}{\quad\quad}
% common reference commands
\newcommand{\eqt}[1]{Eq.~(\ref{#1})}                     % equation
\newcommand{\fig}[1]{Fig.~\ref{#1}}                      % figure
\newcommand{\tbl}[1]{Table~\ref{#1}}                     % table

\newcommand{\ud}{\,\mathrm{d}}
%%%%%%%%%%%%%%%%%%%%%%%%%%%%%%%%%%%%%%%%%%%%%%%%%%%%%%%%%%%%%%%%%%%%

\begin{document}

\begin{center}
{ \Large Answers to Reviewer \#2}
\end{center}

\bigskip

\noindent Ms. Ref. No. JCOMP-D-13-00928\\
Discontinuous Finite Element Solution of the Radiation Diffusion Equation on Arbitrary Polygonal Meshes and Locally Adapted Quadrilateral Grids\\
{\it Journal of Computational Physics},\\

\bigskip
\bigskip

{
\color{blue}
This paper describes a new approach to the solution of the radiation dffusion
equation on arbitrary polygonal meshes and meshes consistent with adaptive
mesh refinement (AMR), using a discontinuous finite element approach with the
symmetric interior penalty technique. My recommendation is that this paper
be published after some minor revisions. The author does a great job making
clear the distinct contribution of the paper and distinguishes this research from
that performed previously.
}

Thank you
\bigskip


{
\color{blue}
There is no mention anywhere in the paper that one downside of discontinuous finite element methods, when compared with continuous finite
element methods, is the much greater number of unknowns. This should
be pointed out, and if there is a computational expense associated with
this increased number unknowns, it should be discussed.
}

Thank you
\bigskip


{
\color{blue}
There is no investigation of the effect of the user supplied parameters in
the symmetric interior penalty (SIP) technique, nor is there any discussion
of whether the results that are obtained are sensitively dependent on these
choices. Do the results obtained require the choices of C and h? made by
the author? Will other choices lead to poorer or better performance?
}


Thank you
\bigskip


{
\color{blue}
A list of typographical, grammatical and notational errors follows. Note:
page numbers are those in the document provided to the reviewer, not those
provided by the author.
}

Thank you for pointing them out and for the thorough review. We have corrected them.

\bigskip







\end{document}

