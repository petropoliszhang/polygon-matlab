\documentclass{article}
%%%%%%%%%%%%%%%%%%%%%%%%%%%%%%%%%%%%%%%%%%%%%%%%%%%%%%%%%%%%%%%%%%%%%%%%%%%%%%%%%%%%%%%%%%%%%%%%%%%%%%%%%%%%%%%%%%%%%%%%%%%%%%%%%%%%%%%%%%%%%%%%%%%%%%%%%%%%%%%%%%%%%%%%%%%%%%%%%%%%%%%%%%%%%%%%%%%%%%%%%%%%%%%%%%%%%%%%%%%%%%%%%%%%%%%%%%%%%%%%%%%%%%%%%%%%
\usepackage{amsmath,amssymb}
% more math
\usepackage{amsfonts}
\usepackage{amssymb}
\usepackage{amstext}
\usepackage{amsbsy}

\usepackage{color}
\newcommand{\mt}[1]{\marginpar{\small #1}}
%%%%%%%%%%%%%%%%%%%%%%%%%%%%%%%%%%%%%%%%%%%%%%%%%%%%%%%%%%%%%%%%%%%%
% new commands
\newcommand{\nc}{\newcommand}
% operators
\renewcommand{\div}{\vec{\nabla}\! \cdot \!}
\newcommand{\grad}{\vec{\nabla}}
% latex shortcuts
\newcommand{\bea}{\begin{eqnarray}}
\newcommand{\eea}{\end{eqnarray}}
\newcommand{\be}{\begin{equation}}
\newcommand{\ee}{\end{equation}}
\newcommand{\bal}{\begin{align}}
\newcommand{\eali}{\end{align}}
\newcommand{\bi}{\begin{itemize}}
\newcommand{\ei}{\end{itemize}}
\newcommand{\ben}{\begin{enumerate}}
\newcommand{\een}{\end{enumerate}}
% DGFEM commands
\newcommand{\jmp}[1]{[\![#1]\!]}                     % jump
\newcommand{\mvl}[1]{\{\!\!\{#1\}\!\!\}}             % mean value
\newcommand{\keff}{\ensuremath{k_{\textit{eff}}}\xspace}
% shortcut for domain notation
\newcommand{\D}{\mathcal{D}}
% vector shortcuts
\newcommand{\vo}{\vec{\Omega}}
\newcommand{\vr}{\vec{r}}
\newcommand{\vn}{\vec{n}}
\newcommand{\vnk}{\vec{\mathbf{n}}}
\newcommand{\vj}{\vec{J}}
% extra space
\newcommand{\qq}{\quad\quad}
% common reference commands
\newcommand{\eqt}[1]{Eq.~(\ref{#1})}                     % equation
\newcommand{\fig}[1]{Fig.~\ref{#1}}                      % figure
\newcommand{\tbl}[1]{Table~\ref{#1}}                     % table

\newcommand{\ud}{\,\mathrm{d}}
%%%%%%%%%%%%%%%%%%%%%%%%%%%%%%%%%%%%%%%%%%%%%%%%%%%%%%%%%%%%%%%%%%%%

\begin{document}

\begin{center}
{ \Large Answers to Reviewer \#2}
\end{center}

\bigskip

\noindent Ms. Ref. No. JCOMP-D-13-00928\\
Discontinuous Finite Element Solution of the Radiation Diffusion Equation on Arbitrary Polygonal Meshes and Locally Adapted Quadrilateral Grids\\
{\it Journal of Computational Physics},\\

\bigskip
\bigskip

{
\color{blue}
This paper describes a new approach to the solution of the radiation dffusion
equation on arbitrary polygonal meshes and meshes consistent with adaptive
mesh refinement (AMR), using a discontinuous finite element approach with the
symmetric interior penalty technique. My recommendation is that this paper
be published after some minor revisions. The author does a great job making
clear the distinct contribution of the paper and distinguishes this research from
that performed previously.
}

Thank you.
\bigskip


{
\color{blue}
There is no mention anywhere in the paper that one downside of discontinuous finite element methods, when compared with continuous finite
element methods, is the much greater number of unknowns. This should
be pointed out, and if there is a computational expense associated with
this increased number unknowns, it should be discussed.
}


This is correct. However, the ultimate goal is to employ this discontinuous FEM discretization for the diffusion equation 
as a synthetic preconditioner for radiation transport solves (typically dicretized using discontinuous FEM). Currently, the 
diffusion synthetic acceleration techniques available for that purpose on arbitrary polygonal grids either result in 
non-symmetric system matrix or require a post-processing treatment to be effective because they rely on a continuous FEM discretizations. Here,
we introduce and test a discontinuous FEM discretization for the diffusion equation and highlight its potential as a diffusion synthetic accelerator 
both in the Introduction and in the Conclusions. 

This PWLD SIP technique for diffusion is now being implemented in a massively C++ radiation transport to be used as a 
synthetic preconditioner for transport solves. In that C++ code, we already have a PWLC (C for continuous)
finite element discretization of the diffusion equation for polygonal grids. We expect to communicate in the near
future DSA results comparing PWLD-SIP versus PWLC (which requires a discontinuous update as a post-processing step to
be an effective preconditioner, as mentioned previously).

\bigskip

{
\color{blue}
There is no investigation of the effect of the user supplied parameters in
the symmetric interior penalty (SIP) technique, nor is there any discussion
of whether the results that are obtained are sensitively dependent on these
choices. Do the results obtained require the choices of C and h? made by
the author? Will other choices lead to poorer or better performance?
}


The constant $C$ needs to be of order 1 for the SIP method to converge; this is described in the SIP references of Kanschat and a SIAM book on DG methods that we have added in the revised manuscript ``Discontinuous Galerkin Methods for Solving Elliptic and Parabolic Equations:  Theory and Implementation'' by Rivi\`ere. When considering arbitrary grids, what really matters is the penalty coefficient that is proportional to $C/h_\perp$.

$h_\perp$ is difficult to 
define for cells of arbitrary shapes (even quadrangles). Thus, we proposed to use the formulae
for the inradius and circumradius of a regular polygon to compute $h^{reg}_\perp$. $h^{reg}_\perp$ of the regular polygon
is well defined but we employ the actual polygon's area and perimeter to
compute the inradius $r$ and circumradius $R$; $h^{reg}_\perp=r+R$ for odd-sided regular polygons and $h^{reg}_\perp=2r$
for even-sided regular polygons. Furthermore, the polygonal meshes generated using a clipped Voronoi diagram of the
quadrangular mesh vertices result in fairly smooth and regular polygons. So, $h_\perp \simeq h^{reg}_\perp$ and thus the standard value of $C=c(p+1)^2=4$ (i.e., $c=1$ here) is retained here since the penalty coefficient is accurately computed. 


However, for extremely skewed Shestakov meshes using quadrangles, $h_\perp $ may become different enough than $h^{reg}_\perp$. In which case, the constant $C$ has to be increased slightly (we used $C=20$ for the worst Shestakov meshes, Fig 8(d) but often a value of $C=4$ or $C=8$ is enough). 

A discussion to this effect has been added to the manuscript.

\bigskip


{
\color{blue}
A list of typographical, grammatical and notational errors follows. Note:
page numbers are those in the document provided to the reviewer, not those
provided by the author.
}

Thank you for pointing them out and for the thorough review. We have corrected them. \\

Regarding Remark \# 26 (``It would be nice for the reader to know that that author has computed the correct $Q(x, y)$. The paper should show the formula for this quantity.''): we use a symbolic toolbox to compute $Q$ from the PDE given an exact solution and to automatically write the corresponding function to be used in the code. It is fairly standard to verify code implementation with the method of manufactured solutions and we hope the readers will also use symbolic toolboxes to generate automatically the source term as needed.

\bigskip







\end{document}

