\documentclass{article}
%%%%%%%%%%%%%%%%%%%%%%%%%%%%%%%%%%%%%%%%%%%%%%%%%%%%%%%%%%%%%%%%%%%%%%%%%%%%%%%%%%%%%%%%%%%%%%%%%%%%%%%%%%%%%%%%%%%%%%%%%%%%%%%%%%%%%%%%%%%%%%%%%%%%%%%%%%%%%%%%%%%%%%%%%%%%%%%%%%%%%%%%%%%%%%%%%%%%%%%%%%%%%%%%%%%%%%%%%%%%%%%%%%%%%%%%%%%%%%%%%%%%%%%%%%%%
\usepackage{amsmath,amssymb}
% more math
\usepackage{amsfonts}
\usepackage{amssymb}
\usepackage{amstext}
\usepackage{amsbsy}

\usepackage{color}
\newcommand{\mt}[1]{\marginpar{\small #1}}
%%%%%%%%%%%%%%%%%%%%%%%%%%%%%%%%%%%%%%%%%%%%%%%%%%%%%%%%%%%%%%%%%%%%
% new commands
\newcommand{\nc}{\newcommand}
% operators
\renewcommand{\div}{\vec{\nabla}\! \cdot \!}
\newcommand{\grad}{\vec{\nabla}}
% latex shortcuts
\newcommand{\bea}{\begin{eqnarray}}
\newcommand{\eea}{\end{eqnarray}}
\newcommand{\be}{\begin{equation}}
\newcommand{\ee}{\end{equation}}
\newcommand{\bal}{\begin{align}}
\newcommand{\eali}{\end{align}}
\newcommand{\bi}{\begin{itemize}}
\newcommand{\ei}{\end{itemize}}
\newcommand{\ben}{\begin{enumerate}}
\newcommand{\een}{\end{enumerate}}
% DGFEM commands
\newcommand{\jmp}[1]{[\![#1]\!]}                     % jump
\newcommand{\mvl}[1]{\{\!\!\{#1\}\!\!\}}             % mean value
\newcommand{\keff}{\ensuremath{k_{\textit{eff}}}\xspace}
% shortcut for domain notation
\newcommand{\D}{\mathcal{D}}
% vector shortcuts
\newcommand{\vo}{\vec{\Omega}}
\newcommand{\vr}{\vec{r}}
\newcommand{\vn}{\vec{n}}
\newcommand{\vnk}{\vec{\mathbf{n}}}
\newcommand{\vj}{\vec{J}}
% extra space
\newcommand{\qq}{\quad\quad}
% common reference commands
\newcommand{\eqt}[1]{Eq.~(\ref{#1})}                     % equation
\newcommand{\fig}[1]{Fig.~\ref{#1}}                      % figure
\newcommand{\tbl}[1]{Table~\ref{#1}}                     % table

\newcommand{\ud}{\,\mathrm{d}}

\newcommand{\tcr}[1]{\textcolor{red}{#1}}
%%%%%%%%%%%%%%%%%%%%%%%%%%%%%%%%%%%%%%%%%%%%%%%%%%%%%%%%%%%%%%%%%%%%

\begin{document}

\begin{center}
{ \Large Answers to Reviewer \#1}
\end{center}

\bigskip

\noindent Ms. Ref. No. JCOMP-D-13-00928\\
Discontinuous Finite Element Solution of the Radiation Diffusion Equation on Arbitrary Polygonal Meshes and Locally Adapted Quadrilateral Grids\\
{\it Journal of Computational Physics},\\

\bigskip
\bigskip

{
\color{blue}
This paper is a very high quality paper for the Journal of Computational Physics and should be
accepted for publication without reservation. One of the main strengths of this paper is its very thorough
review of spatial discretization technology for radiation diffusion. It is well written, and includes a
comprehensive set of test problems applied to the new discretization proposed.
}

Thank you for your thorough review and appreciation of our work. 
\bigskip


{
\color{blue}
I have some suggestions for improvement, but as a reviewer, I do not require these additions to the paper
for publication.
}


Thank you. Even though you do not require these additions for publication, I take it at heart to publish the 
possible papers and I will include as many suggestions for improvement as possible in the revised version of 
this manuscript.
\bigskip



{
\color{blue}
On the list or prior works, a reference or references should be included for each item in the list. For the
first two items, no references are included.
}


In this short enumeration of prior works, our intent was not to add new references but to recall
the various techniques we have just reviewed in the above paragraphs. However, we have forgotten to 
discuss FV techniques for polygonal meshes. This is now corrected and a sentence has been added in the 
above literature review. Mimetic FD were already described and reviewed above so we do not think we are missing 
references for this type of discretization techniques. 
\bigskip

{
\color{blue}
The derivation of the discretization is complicated enough to warrant an appendix which goes through
the details of the mathematics (e.g. how to get to Eq. 3 and Eq. 8), like one would find in a thesis or
dissertation. I am a proponent of including these detailed types of explanations in papers because I think
they enhance the read-ability of the paper; however, some disagree with this philosophy.
}


The articles we cite ``Unified Analysis of Discontinuous Galerkin Methods for Elliptic Problems'' (Douglas et. al.),
the habilitation thesis ``Discontinuous Galerkin Methods for Viscous Incompressible Flow'' by Kanschat and a SIAM book on DG 
methods that we have added in the revised manuscript ``Discontinuous Galerkin Methods for Solving Elliptic and Parabolic Equations: Theory and Implementation'' by Rivere\'ere are resources that provide an excellent description of the SIP method for elliptic problems (along with many DG variants). 
\bigskip

{
\color{blue}
The author could state more clearly how this work is distinct from previous work. The author mentions
that others have applied a Discontinuous method to diffusion operators. How is the proposed method
different? In section 3, the author writes: "Many variants of such discontinuous discretization methods
exist for diffuion problems…" It may be a good idea to say, the discretization in this paper is different
because…
}
The DG technique for the radiation diffusion equation is the standard Symmetric Interior Penalty technique. 
We chose it because it yields a symmetric linear system that can be efficiently solved using
a preconditioned conjugate gradient technique. The novelty is the use SIP on polygonal meshes using PWLD basis functions
(and to perform  AMR with PWLD).

\bigskip

{
\color{blue}
I am curious as to why the statement was made that "we have opted to use two loops: one over the
elements…" Did this have an impact of computational performance? It seems like the reason why was never
addressed.
}

The DG bilinear form, Eq.(8), may look a bit complex. However, it only contains volume (surface in 2D) integrals 
and face (edge in 2D) integrals. The volume integrals are split on each element (thus we loop over the elements). 
Each integral in the volume of a given element only contribute entries in the global matrix for the unknowns present 
in that element. The face integrals are split over the faces (thus we loop over the set of interior faces). Each face integral
contributes entries in the global matrix for the  unknowns present in the two elements separating by that face. 
Thus, the manner in which cell/face information is accessed and used is different. It felt like coding would be clearer if  our implementation would mimic the various terms found in the bilinear form of Eq.(8).

\bigskip

{
\color{blue}
For the discussion at the top of page 9 about C and h$_\perp$ a picture may be helpful, especially when
describing the inradius and circumradius for polygons.
}


\tcr{Thank you}
\bigskip


{
\color{blue}
I am curious if the author has tried standard linear basis functions (u(x,y) = a + bx + cy) on
quadrilaterals to see if this works for the DFEM type diffusion discretization.
}

One can, of course, use linear basis functions on quadrilaterals. However, the ultimate goal is to employ
the diffusion equation as a preconditioner to a radiation transport solver. In the thick diffusion limit,
employing linear basis functions on quadrilaterals leads to a locking of the transport solution to the 
wrong answer. In reactor problems, where the thick diffusion limit is not a relevant regime, researcher
may prefer to use linear basis functions (a+bx+cy+dz) on hexadredal cells to reduce the number of unknowns
without drastically reducing accuracy.

\bigskip

{
\color{blue}
What code was this method implemented in? Was it just a test code? Was it MATLAB? Were there any
parallel runs? What linear solver package was used, if any?
}
The test bed code is MATLAB. We use both the built-in optimized LU package but also a conjugate gradient solve
preconditioned with an algebraic multigrid method. The PWLD SIP technique for diffusion is now being 
implement in a massively C++ radiation transport to be used either as a standalone solver or a 
synthetic preconditioner. 



\bigskip


{
\color{blue}
I think a potential weakness of the paper is that the author does not compare the new method to any of
the existing methods. If this is not hard to do, these results would be very interesting to include in the
paper. Additionally, these results could be included in a future paper or conference proceedings.
}


Thank you
\bigskip


{
\color{blue}
For the linear test problem results, what are the iteration counts for the linear solve as the mesh is
distorted? Was the iteration count a lot different for the different mesh types? For quads vs. polygons?
}


Thank you
\bigskip


{
\color{blue}
A reference for computing bounded Voronoi diagrams is a good idea.
}


Thank you
\bigskip


{
\color{blue}
For the convergence plots, I think the terminology is a bit inconsistent. In the text the author writes
in terms of dof. In the plots, he uses number of unknowns. This could be a bit more consistent, but is a
minor detail.
}


Thank you
\bigskip


{
\color{blue}
In general, an understanding of how resistant this method is to negative solutions would be another
interesting thing to study, especially in a time dependent case, with a delta-function like source. I would
anticipate that this method would perform a lot better than the CFEMs.
}


Thank you
\bigskip



{
\color{blue}
Does the author anticipate a straight-forward application to RZ geometry and 3D geometry? Will the
method recover spherical symmetry in RZ geometry. Brunner, et. Al. have observed an issue with this for RZ
geometry. See T.A. Brunner…, "Perserving Spherical Symmetry in Axisymmetric Coordinates for Diffusion,"
in Proc. International Conference on Mathematics and Computational Methods Applied to Nuclear Science and
Engineering, May 5-9, 2013, Sun Valley, ID (2013), CD-ROM
}


Thank you
\bigskip


\end{document}

