\documentstyle[11pt]{letter}

%%%%%% Letter Size Setup %%%%%%%%%%%%%%%%%%%%%%%%%%%%%%%%%%%%%%%%%%%%%%%%%%
%        \addtolength{\textwidth}{2.5cm}     %% For longer or shorter text width
        \addtolength{\topmargin}{-1.5cm}    %% For more or less top margin
       \addtolength{\textheight}{7cm}    %% For longer or shorter textheight
%        \addtolength{\oddsidemargin}{-1.25cm} %% For odd side margin (twoside)
                                            %% or margin (oneside)

\address{Jean Ragusa\\ 
Department of Nuclear Engineering \\
Texas A\&M University\\
College Station, TX 77843-3133, USA\\
phone: (979) 862 2033\\
e-mail: jean.ragusa@tamu.edu \vspace{0.5cm}}

%%%%%% The Signature  and Date %%%%%%%%%%%%%%%%%%%%%%%%%%%%%%%%%%%%%%%%%%%%

\signature{\vspace{-1.25cm}Jean Ragusa}   


\begin{document}

\begin{letter}{Professor William Martin\\
    Editor,\\
    Journal of Computational Physics}
\date{\today}
%%%%%% More vertical space can be added here %%%%%%%%%%%%%%%%%%%%%%%%%%%%%%
%         \vspace{3.0cm}

\opening{Dear Professor Martin,}
         \vspace{0.25cm}
%%%%%% More vertical space can be added here %%%%%%%%%%%%%%%%%%%%%%%%%%%%%%

Please find attached a copy of our manuscript titled ``Discontinuous Finite Element Solution of the Radiation Diffusion Equation on Arbitrary Polygonal Meshes and Locally Adapted Quadrilateral Grids'' for submission to the {\em Journal of Computational Physics}. 

In this paper, we propose a piecewise linear \underline{\it discontinuous} finite element discretization of the standard diffusion equation for arbitrary polygonal grids. 


This work follows closely prior works by Jim Morel (LANL, now TAMU), Mikhail Shashkov (LANL), Teresa Bailey (LLNL), Marvin Adams (TAMU), and Todd Palmer (OSU).

Our work is based on the standard diffusion form and uses the symmetric interior penalty technique, which yields a 
symmetric positive definite linear system matrix. A preconditioned conjugate gradient algorithm is employed to solve the linear system. 
Piecewise linear approximations also allow a straightforward implementation of local mesh adaptation by allowing unrefined cells to
be interpreted as polygons with an increased number of vertices.
Several test cases, taken from the literature on discretizations of the radiation diffusion, are presented: random, sinusoidal, Shestakov, 
and Z meshes are used. The last numerical example demonstrates the application of the PWLD discretization to adaptive mesh refinement.

Employing a DFEM approximation to solve the diffusion equation is not commonly done (especially on polygonal grids), but this will allow us to use proven DFEM-based DSA preconditioners to tackle radiation transport problems on polygonal grids in the future.



Thank you for considering this manuscript for publication in JCP.

\vspace{-0.25cm}


%%%%%% More vertical space can be added here %%%%%%%%%%%%%%%%%%%%%%%%%%%%%%

%%%%%%% The Closing %%%%%%%%%%%%%%%%%%%%%%%%%%%%%%%%%%%%%%%%%%%%%%%%%%%%%%%
\closing{Best regards, }

\end{letter}
\end{document}

